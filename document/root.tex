\documentclass{article}
\usepackage[a4paper, margin=1in]{geometry}
\usepackage{isabelle,isabellesym}

% this should be the last package used
\usepackage{pdfsetup}

% urls in roman style, theory text in math-similar italics
\urlstyle{rm}
\isabellestyle{it}

\begin{document}

\title{Registers}
\author{Dominique Unruh}
\maketitle

\begin{abstract}
  A formalization of the theory of quantum and classical registers as
  developed by Unruh \cite{unruh21registers}. In a nutshell, a
  register refers to a part of a larger memory or system that can be
  accessed independently.  Registers can be constructed from other
  registers and several (compatible) registers can be composed. For
  more details, see \cite{unruh21registers}. This formalization
  develops both the generic theory of registers as well as specific
  instantiations for classical and quantum registers.
\end{abstract}

\tableofcontents

% sane default for proof documents
\parindent 0pt\parskip 0.5ex

% generated text of all theories
\input{session}

\bibliographystyle{alpha}
\bibliography{root}

\end{document}

%%% Local Variables:
%%% mode: latex
%%% TeX-master: t
%%% End:
